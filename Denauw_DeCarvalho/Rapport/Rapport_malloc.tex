\documentclass[10pt,a4paper]{article}
\usepackage{inputenc}
\usepackage[T1]{fontenc}
\usepackage[french]{babel}
\usepackage{fontspec}
\usepackage{graphicx}


\begin{document}

\author{Denauw Antoine\\ De Carvalho Borges Antonio}
\title{SINF1252 - Rapport - Implémentation de malloc et free}
\maketitle

\tableofcontents

%Vous devez écrire un rapport au format PDF d’une ou deux pages, contenant les explications de votre implémentation, les difficultés rencontrées, les cas couverts par vos tests unitaires ainsi qu’une figure montrant les résultats d’une mesure de performance de votre implémentation.

\newpage

\section{Explications de l'implémentation}

\paragraph{Aide de Tonio}


\section{Difficultés rencontrées}

\paragraph{}Les difficultés rencontrées ont été assez nombreuses, tout d'abord, la visualisation du problème était assez complexe. Nous avons eu un peu de mal à comprendre précisément ce qui était demander et nous avons aussi remarqué qu'il était essentiel de comprendre ce que faisais malloc dans les moindre détails pour pouvoir en créer un équivalent.

\paragraph{}Une fois compris clairement les consignes et malloc, nous nous sommes attaqués à la structure de tout les fichiers que devait regrouper notre programme car c'est la première fois que nous utilisions CUnit mais aussi un Makefile qui devait être un peu plus complet que dans le projet précédent.

\paragraph{}Pour l'implémentation de malloc, nous nous sommes réunis en salle infos et nous avons donné nos idées pour les comparer et ensuite nous avons codé pas à pas la fonction malloc. Nous avons eu quelques difficultés au niveau des pointeurs mais aussi pour gérer la mémoire, nous entendons par là les cas où il n'y avait plus de place dans le bloc mémoire demandé que nous avons alloué via la fonction sbrk. Les cas d'appel répété à malloc car nous devions regarder si, dans les blocs qui ont été alloué, et, qui ont été free par après, étaient assez grand pour accueillir l'espace demandé par l'appel suivant de malloc.

\paragraph{FREE}


\paragraph{CALLOC}

\paragraph{RAPPORT}


\section{Cas couverts par nos tests unitaires}

\paragraph{SOON}


\section{Performance de notre implémentation}

\paragraph{FIN d'implémentation}


%\includegraphics[scale=•]{•}



\end{document}